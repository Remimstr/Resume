%%%%%%%%%%%%%%%%%%%%%%%%%%%%%%%%%%%%%%%%%
% Twenty Seconds Resume/CV
% LaTeX Template
% Version 1.0 (14/7/16)
%
% This template has been downloaded from:
% http://www.LaTeXTemplates.com
%
% Original author:
% Carmine Spagnuolo (cspagnuolo@unisa.it) with major modifications by 
% Vel (vel@LaTeXTemplates.com)
%
% License:
% The MIT License (see included LICENSE file)
%
%%%%%%%%%%%%%%%%%%%%%%%%%%%%%%%%%%%%%%%%%

%----------------------------------------------------------------------------------------
%	PACKAGES AND OTHER DOCUMENT CONFIGURATIONS
%----------------------------------------------------------------------------------------

\documentclass[letterpaper]{twentysecondcv} % a4paper for A4

%----------------------------------------------------------------------------------------
%	 PERSONAL INFORMATION
%----------------------------------------------------------------------------------------

% If you don't need one or more of the below, just remove the content leaving the command, e.g. \cvnumberphone{}

\profilepic{alice.jpeg} % Profile picture

\cvname{Remi Marchand} % Your name
\cvjobtitle{3B Honours Biology,} % Job title/career
\cvsubtitle{Computer Science Minor}

\cvdate{}
\cvaddress{Richmond Hill, ON} % Short address/location, use \newline if more than 1 line is required
\cvsite{https://github.com/Remimstr} % Personal website
\cvnumberphone{(647) 995-3645} % Phone number
\cvmail{remimarchand@mail.com} % Email address

%----------------------------------------------------------------------------------------

\begin{document}

%----------------------------------------------------------------------------------------
%	 ABOUT ME
%----------------------------------------------------------------------------------------

\aboutme{\begin{itemize}
		\renewcommand\labelitemi{--}
		\item{Experience solving complex problems in biology using computational approaches}
		\item{Excited by new challenges; I love being thrown into the "deep end" of a project}
		\item{Passionate being part of a team that accomplishes ambitious goals such as whole genome cancer analysis}
		\item{Coursework and self-taught in Python, Bash, Java; Some PHP, C, R, and HTML}
		\item{Familiar writing clean, modular code, and using VCS such as Git}
		\end{itemize}
}

%----------------------------------------------------------------------------------------
%	 SKILLS
%----------------------------------------------------------------------------------------

% Skill bar section, each skill must have a value between 0 an 6 (float)
\skills{{Web Development/1.5},
		{VCS - Git/3},
		{Data Science/4.5},
		{Python/5.4},
		{Java/1.8},
		{C/3.7}}

%------------------------------------------------


% Skill text section, each skill must have a value between 0 an 6
\skillstext{{lovely/4},{narcissistic/3}}

%----------------------------------------------------------------------------------------

\makeprofile % Print the sidebar

%----------------------------------------------------------------------------------------
%	 INTERESTS
%----------------------------------------------------------------------------------------

%\section{Profile}
%
%\begin{itemize}
%	\item{Experience solving big problems in biology using computational approaches}
%	\item{Ability to consistently deliver results ahead of schedule and under pressure}
%	\item{Passion for being part of a team that accomplishes ambitious goals such as whole genome cancer analysis and database }
%	\item{Coursework and self-taught in Python, Bash, Java; Some PHP, C, R, and HTML}
%	\item{Familiar writing clean, modular code, and using VCS such as Git}
%\end{itemize}
%\vspace{\parsep}%


%----------------------------------------------------------------------------------------
%	 WORK EXPERIENCE
%----------------------------------------------------------------------------------------

\section{Work Experience}

\begin{twenty} % Environment for a list with descriptions
	\twentyitem{Fall 2016}{Bioinformatics Support}{Neufeld Lab, Waterloo}{
		\begin{itemize}
			\item{Performed maintenance and support for servers and backups}
			\item{Created and modified scripts as required by teams of researchers for their specific projects}
 			\item{Prioritized tasks and documented work carefully through github}
		\end{itemize}}
	\twentyitem{Spring 2016}{Bioinformatics Research Assistant}{National Microbiology Laboratory, Guelph}{
		\begin{itemize}
			\item{Created tools in python for the curation of > 63,000 Salmonella samples resulting in a comprehensive collection of metadata to be used for analysis}
			\item{Implemented automation using custom spellchecker and controlled list modules reducing manual curation required}
			\item{Code at: \href{https://github.com/Remimstr/Standardize_Metadata}{\color{blue}https://github.com/Remimstr/Standardize\_Metadata}}
		\end{itemize}}
	\twentyitem{Fall 2015}{Computational Cancer Genomics}{SickKids Hospital, Toronto}{
		\begin{itemize}
			\item{Worked as part of a team to characterize paediatric cancer samples using a high performance computing cluster}
			\item{Reduced the number of false positive translocations called by our pipeline 10-fold; from an average of 150 to an average of 16}
			\item{Designed and wrote a novel tool in python that used translocation events to predict tumor evolution}
		\end{itemize}}
	\twentyitem{Winter 2015}{Instructional Support Assistant}{Faculty of Math, University of Waterloo}{
		\begin{itemize}
			\item{}
			\item{Conveyed my knowledge and passion for computer science to students}
			\item{Designed and wrote a novel tool in python that used translocation events to predict tumor evolution}
		\end{itemize}}
	%\twentyitem{<dates>}{<title>}{<location>}{<description>}
\end{twenty}


%----------------------------------------------------------------------------------------
%	 EDUCATION
%----------------------------------------------------------------------------------------

\section{Education}

\begin{twenty} % Environment for a list with descriptions
	\twentyitem{since 1865}{Ph.D. {\normalfont candidate in Computer Science}}{Wonderland}{\emph{A Quantified Theory of Social Cohesion.}}
	\twentyitem{1863-1865}{M.Sc. magna cum laude}{Wonderland}{Majoring in Computer Science}
	\twentyitem{1861-1863}{B.Sc. magna cum laude}{Wonderland}{Majoring in Computer Science}
	\twentyitem{1856-1861}{High school}{Wonderland}{Specializing in mathematics and physics.}
	%\twentyitem{<dates>}{<title>}{<location>}{<description>}
\end{twenty}

%----------------------------------------------------------------------------------------
%	 PUBLICATIONS
%----------------------------------------------------------------------------------------

%%%%%%%%%TWENTY LIST SHORTITEMS%%%%%%%%%%%%%%
%%% Two arguments: date; title/description %%%%%%%%%%
\section{Publications}

\begin{twentyshort} % Environment for a short list with no descriptions
	\twentyitemshort{1865}{Chapter One, Down the Rabbit Hole.}
	\twentyitemshort{1865}{Chapter Two, The Pool of Tears.}
	\twentyitemshort{1865}{Chapter Three,  The Caucus Race and a Long Tale.}
	\twentyitemshort{1865}{Chapter Four,  The Rabbit Sends a Little Bill.}
	\twentyitemshort{1865}{Chapter Five,  Advice from a Caterpillar.}
	%\twentyitemshort{<dates>}{<title/description>}
\end{twentyshort}

%----------------------------------------------------------------------------------------
%	 AWARDS
%----------------------------------------------------------------------------------------

\section{Awards}

\begin{twentyshort} % Environment for a short list with no descriptions
	\twentyitemshort{1987}{All-Time Best Fantasy Novel.}
	\twentyitemshort{1998}{All-Time Best Fantasy Novel before 1990.}
	%\twentyitemshort{<dates>}{<title/description>}
\end{twentyshort}

%----------------------------------------------------------------------------------------
%	 EXPERIENCE
%----------------------------------------------------------------------------------------

\section{Experience}

\begin{twenty} % Environment for a list with descriptions
	\twentyitem{1900}{Alice in Wonderland-The Circra (1900's) Silent Film.}{Film}{The first Alice on film was over a hundred years ago.}
	\twentyitem{1933}{Alice in Wonderland 1933 version.}{Film}{This film stars Ethel griffies and Charlotte Henry. It was a box office flop when it was released.}
	\twentyitem{1951}{Disney Film.}{Film}{Walt Disney brings Lewis Carroll's fantasy story to life in this well done animated classic. Even though many elements from the book were dropped, such as the duchess with the baby pig and mock turtle, this version is without a doubt the most famous Alice adaption made.}
	%\twentyitem{<dates>}{<title>}{<location>}{<description>}
\end{twenty}

%----------------------------------------------------------------------------------------
%	 OTHER INFORMATION
%----------------------------------------------------------------------------------------

\section{Other Information}

\subsection{Review}

Alice approaches Wonderland as an anthropologist, but maintains a strong sense of noblesse oblige that comes with her class status. She has confidence in her social position, education, and the Victorian virtue of good manners. Alice has a feeling of entitlement, particularly when comparing herself to Mabel, whom she declares has a ``poky little house," and no toys. Additionally, she flaunts her limited information base with anyone who will listen and becomes increasingly obsessed with the importance of good manners as she deals with the rude creatures of Wonderland. Alice maintains a superior attitude and behaves with solicitous indulgence toward those she believes are less privileged.

%----------------------------------------------------------------------------------------
%	 SECOND PAGE EXAMPLE
%----------------------------------------------------------------------------------------

%\newpage % Start a new page

%\makeprofile % Print the sidebar

%\section{other information}

%\subsection{Review}

%Alice approaches Wonderland as an anthropologist, but maintains a strong sense of noblesse oblige that comes with her class status. She has confidence in her social position, education, and the Victorian virtue of good manners. Alice has a feeling of entitlement, particularly when comparing herself to Mabel, whom she declares has a ``poky little house," and no toys. Additionally, she flaunts her limited information base with anyone who will listen and becomes increasingly obsessed with the importance of good manners as she deals with the rude creatures of Wonderland. Alice maintains a superior attitude and behaves with solicitous indulgence toward those she believes are less privileged.

%\section{other information}

%\subsection{Review}

%Alice approaches Wonderland as an anthropologist, but maintains a strong sense of noblesse oblige that comes with her class status. She has confidence in her social position, education, and the Victorian virtue of good manners. Alice has a feeling of entitlement, particularly when comparing herself to Mabel, whom she declares has a ``poky little house," and no toys. Additionally, she flaunts her limited information base with anyone who will listen and becomes increasingly obsessed with the importance of good manners as she deals with the rude creatures of Wonderland. Alice maintains a superior attitude and behaves with solicitous indulgence toward those she believes are less privileged.

%----------------------------------------------------------------------------------------

\end{document} 
